\chapter{Introduction}

\section{Contexte}

\paragraph{}
Dans le cadre du PJI (Projet Individuel) en fin de master 1, j'ai choisi de travailler sous la tutelle de monsieur LUXEY-BITRI sur le sujet des communications réseaux dites pair à pair.
Avec peu de connaissances en réseaux, j'y ai vu l'occasion de découvrir un domaine qui m'était inconnu et de me confronter à un sujet de recherche pour l'instant peu exploré mais 
avec un potentiel d'application très important. En effet, si aujourd'hui les communications réseaux sont majoritairement centralisées, on voit émerger de plus en plus de réseaux
dits décentralisés. 

\paragraph{}
Les réseaux décentralisés, également connus sous le nom de réseaux pair-à-pair (P2P), offrent une approche alternative à la communication et à l'échange de données. Contrairement aux réseaux centralisés, 
où les données transitent par un point central tel qu'un serveur, les réseaux P2P permettent aux utilisateurs de communiquer directement entre eux, en utilisant leurs propres ressources 
(telles que la puissance de calcul et la bande passante) pour faciliter les échanges.

\paragraph{}
Ce modèle de communication présente plusieurs avantages, notamment une meilleure résilience face aux pannes, une répartition de la charge plus équilibrée et une réduction de la dépendance 
à l'égard d'entités centrales. Les réseaux P2P sont largement utilisés dans divers domaines, tels que le partage de fichiers (comme BitTorrent), la messagerie instantanée (comme Skype) 
et les systèmes de paiement (comme Bitcoin).

\newpage

\paragraph{}
J'ai choisi ce sujet car la majorité de mon travail personnel consiste à développer des applications web basées sur des architectures centralisées. Je souhaitais donc découvrir un nouveau
domaine et me confronter à de nouvelles problématiques. De plus, les réseaux pair à pair sont un sujet de recherche très actuel et qui a un fort potentiel d'application. En effet,
les réseaux pair à pair sont de plus en plus utilisés dans le domaine des communications réseaux. On peut citer par exemple le protocole WebRTC qui permet la communication audio et vidéo
entre deux clients sans passer par un serveur.


\section{Problématique}

\paragraph{}
Quelles sont les solutions possible pour mettre en place une communication textuelle temps réel entre deux clients dans un réseau pair à pair ?

\section{Objectifs}

\paragraph{}
L'objectif de ce projet est l'étude des mécanismes permettant le "pair à pair" dans le web moderne grace à l'implémentation de prototypes permettant la communication textuelle entre deux clients.

\section{Plan}

\begin{enumerate}
\item Dans un premier temps, afin de mieux comprendre le fonctionnement des réseaux pair à pair, nous allons étudier et présenter brièvement les différents les solutions existantes

\item Dans un second temps, nous étudierons les différentes façons de mettre en place une communication textuelle temps réel entre deux clients dans un réseau pair à pair. Nous verrons
qu'il existe plusieurs solutions possibles telles que l'utilisation de sockets, l'utilisation de WebRTC ou encore l'utilisation de WebSockets.

\item Dans un troisième temps, présenterons la l'alternative avec WebRTC que nous avons implémenter. Nous expliquerons les raisons de ce choix et nous détaillerons l'implémentation de
notre solution. Avant de tenter la réalisation de cette même solution via des sockets. Nous verrons dans cette partie les difficultés rencontrées, les solutions apportées et les
limites de notre implémentation. 

\item Enfin, nous conclurons sur les résultats obtenus et nous évoquerons les perspectives d'amélioration de notre solution.

\end{enumerate}