\chapter{Conclusion}

\paragraph{}
En conclusion, ce document a souligné la problématique de l'établissement d'une connexion pair à pair en présence de dispositifs de traduction d'adresses réseau (NAT). Les NAT, bien qu'utiles pour partager une adresse IP publique et protéger 
les dispositifs internes, limitent la connectivité directe entre les clients.

\paragraph{}
Pour contourner ces limitations, il est nécessaire de recourir à des solutions telles que WebRTC et libp2p. WebRTC, conçu pour les communications en temps réel, et libp2p, une bibliothèque modulaire pour la communication pair à pair, 
offrent des mécanismes pour établir des connexions directes malgré la présence de dispositifs NAT.

\paragraph{}
En utilisant ces approches, les clients peuvent s'échanger les adresses multiadresses nécessaires pour établir la connexion et communiquer directement, sans dépendre exclusivement d'un tiers relai. Cependant, il est important de prendre
 en compte les besoins spécifiques de chaque application pour choisir la solution la mieux adaptée.

\paragraph{}
En explorant et en implémentant ces solutions, il devient possible de surmonter les limitations des dispositifs NAT et de faciliter l'établissement de connexions pair à pair. Cela ouvre la voie à une communication directe et efficace entre les clients, 
améliorant ainsi les possibilités de collaboration et d'échange de données dans divers scénarios d'application.