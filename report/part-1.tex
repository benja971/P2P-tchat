\chapter{Contexte et Motivation}

\paragraph{}
Dans le domaine des communications réseaux pair à pair, il est souvent nécessaire d'établir une communication directe entre deux clients. Cependant, certains défis se 
posent lors de la mise en place d'une telle communication, notamment en présence de dispositifs de traduction d'adresses réseau (NAT) qui limitent la connectivité directe entre les clients.

\section{Problèmes liés aux dispositifs de traduction d'adresses réseau (le NAT)}

\paragraph{}
Les dispositifs de traduction d'adresses réseau (NAT) sont couramment utilisés pour partager une seule adresse IP publique entre plusieurs appareils d'un réseau local. Les NAT 
peuvent être trouvés dans de nombreux environnements, tels que les réseaux domestiques, les réseaux d'entreprises et même les réseaux mobiles.

\paragraph{}
Lorsqu'un client se trouve derrière un NAT, il reçoit une adresse IP privée non routable, qui n'est pas accessible depuis Internet. Cela rend difficile l'établissement 
d'une communication directe avec d'autres clients qui se trouvent également derrière des NAT.

\newpage
\section{La nécessité d'un tiers pour faciliter la communication}

\paragraph{}
Dans le contexte des communications pair à pair, il devient nécessaire d'avoir un tiers (un serveur intermédiaire) pour faciliter la communication entre les clients. 
Ce tiers agit comme un relai en permettant aux clients de s'échanger des données même s'ils ne peuvent pas établir une connexion directe en raison des NAT.

\paragraph{}
Le relai joue un rôle crucial en tant qu'entremetteur dans la communication entre les clients. Il permet aux clients de s'enregistrer et de découvrir les autres clients disponibles. 
Lorsqu'un client souhaite établir une communication avec un autre client, le relai facilite l'établissement de la connexion en relayant les messages entre les clients.

\section{Solutions existantes utilisant WebSockets et WebRTC}

\paragraph{}
Dans ce contexte, différentes solutions ont été développées pour permettre une communication efficace entre les clients. Deux technologies largement utilisées sont les WebSockets et WebRTC.

\paragraph{}
Les WebSockets offrent une communication bidirectionnelle et persistante entre un navigateur Web et un serveur WebSocket. Ils permettent une communication en temps réel, avec une latence 
réduite et une meilleure efficacité par rapport aux requêtes HTTP traditionnelles.

\paragraph{}
WebRTC, quant à lui, permet une communication peer-to-peer directe entre les navigateurs Web. Il prend en charge la communication audio, vidéo et de données en temps réel. 
WebRTC permet aux clients d'établir des connexions directes, contournant ainsi les NAT et réduisant la latence.

\paragraph{}
Ces technologies offrent des solutions puissantes pour la communication en temps réel dans un réseau pair à pair. Cependant, leur utilisation nécessite la mise en place de mécanismes de 
signalisation et de relais pour l'établissement des connexions et la transmission des données entre les clients.


\section{Architecture générale d'un système réseau avec NAT}

Dans un système réseau avec dispositifs de traduction d'adresses réseau (NAT), l'architecture générale implique plusieurs composants qui interagissent pour permettre la communication entre les clients. 
Voici les principaux composants et leur rôle :

\paragraph{}
\textbf{Clients} : Les clients sont les dispositifs ou les applications qui souhaitent communiquer entre eux. Ils peuvent être des ordinateurs, des téléphones, des tablettes ou tout autre appareil 
connecté au réseau. Les clients peuvent être situés derrière des NAT, ce qui limite leur capacité à établir une communication directe avec d'autres clients. 

\paragraph{}
\textbf{Serveur relai} : Le serveur relai agit en tant qu'entremetteur dans la communication entre les clients. Lorsqu'un client souhaite communiquer avec un autre client, il s'enregistre auprès 
du serveur relai et communique ses informations de connexion. Le serveur relai joue un rôle clé dans la découverte des clients disponibles et facilite l'établissement de la communication en relayant les messages entre les clients.

\paragraph{}
\textbf{Mécanisme de signalisation} : Le mécanisme de signalisation est utilisé pour l'échange d'informations entre les clients et le serveur relai. Il permet aux clients de s'enregistrer auprès du 
serveur relai, de découvrir les autres clients disponibles et d'échanger les informations nécessaires pour établir une connexion. La signalisation peut se faire via des protocoles tels que le protocole 
HTTP, les WebSockets ou d'autres protocoles personnalisés.

\paragraph{}
\textbf{Mécanisme de traversée NAT} : Étant donné que les clients peuvent être situés derrière des NAT, un mécanisme de traversée NAT est nécessaire pour permettre l'établissement de la communication directe entre 
les clients. Ce mécanisme utilise des techniques telles que le protocole de traversée NAT (NAT traversal protocol) ou les techniques de trous de poinçonnage (hole punching) pour contourner les limitations des NAT 
et permettre aux clients de se connecter directement.

\paragraph{}
\textbf{Protocoles de communication} : Une fois que la communication directe est établie entre les clients, les protocoles de communication tels que les WebSockets ou WebRTC peuvent être utilisés pour la 
transmission des données. Ces protocoles permettent une communication bidirectionnelle en temps réel, avec une latence réduite et une efficacité améliorée par rapport aux protocoles traditionnels.

\paragraph{}
En combinant ces composants et mécanismes, l'architecture générale d'un système réseau avec NAT permet aux clients de surmonter les limitations de connectivité causées par les dispositifs de 
traduction d'adresses réseau. Cela permet aux clients de communiquer efficacement même s'ils sont situés derrière des NAT, en utilisant des serveurs relais et des mécanismes de signalisation et de traversée NAT appropriés.




