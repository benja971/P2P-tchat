\chapter{Solutions existantes}

Dans le domaine des communications réseaux pair à pair, plusieurs solutions ont déjà été développées pour établir une communication entre deux clients. 
Dans cette partie, nous présenterons brièvement quelques-unes de ces solutions.

\section{Les technologies de communication Web}

Avec l'avènement du Web, de nouvelles technologies de communication ont émergé pour permettre des interactions directes entre les utilisateurs. Parmi celles-ci, nous pouvons citer les WebSockets et WebRTC.

\subsection{WebSockets}

\paragraph{}
Les WebSockets sont une technologie de communication bidirectionnelle qui permet aux navigateurs Web d'établir une connexion persistante avec un serveur. Contrairement aux requêtes HTTP traditionnelles, 
qui suivent le modèle de requête-réponse, les WebSockets permettent une communication en temps réel, où les données peuvent être transmises dans les deux sens de manière asynchrone.

\paragraph{}
Les WebSockets sont largement utilisés dans les applications nécessitant une communication instantanée, telles que les chats en temps réel, les tableaux de bord de suivi des données en temps réel, 
les jeux en ligne, etc. Ils offrent une latence réduite et une meilleure efficacité en éliminant le besoin de requêtes HTTP fréquentes pour récupérer les mises à jour des données.

\newpage

\paragraph{}
En utilisant les WebSockets, les clients peuvent établir une connexion directe avec un serveur WebSocket, et une fois la connexion établie, ils peuvent échanger des messages sous forme de flux de données. 
Cette technologie permet une communication bidirectionnelle et simultanée, ce qui en fait une solution efficace pour la communication textuelle temps réel entre deux clients dans un réseau pair à pair.


\subsection{WebRTC}

\paragraph{}
WebRTC (Web Real-Time Communication) est une technologie qui permet aux navigateurs Web d'établir des connexions peer-to-peer directes pour la communication audio, vidéo et de données en temps réel. 
WebRTC utilise une combinaison de protocoles et de codecs pour faciliter la communication directe entre les navigateurs, sans passer par un serveur intermédiaire.

\paragraph{}
WebRTC a ouvert de nouvelles possibilités pour des applications telles que les appels audio et vidéo en temps réel, les conférences Web, les partages d'écran et bien plus encore. Il permet aux utilisateurs 
d'interagir directement sans avoir à installer de plugins ou de logiciels tiers.

\paragraph{}
Dans le contexte de la communication textuelle temps réel dans un réseau pair à pair, WebRTC peut être utilisé pour établir une connexion directe entre deux clients, permettant ainsi une communication en temps réel. 
Les clients peuvent échanger des messages textuels en utilisant la fonctionnalité de transfert de données offerte par WebRTC.

\paragraph{}
L'avantage de WebRTC réside dans sa capacité à établir des connexions directes entre les clients, ce qui réduit la latence et offre une expérience de communication plus fluide. Cependant, il convient de noter que 
l'utilisation de WebRTC nécessite la mise en place de mécanismes de signalisation pour l'établissement de la connexion initiale entre les clients, ainsi que des aspects de sécurité pour protéger la confidentialité des données échangées.



