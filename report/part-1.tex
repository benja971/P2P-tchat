\chapter{Introduction}

\section{Contexte}

\paragraph{}

Dans le cadre du PJI (Projet Individuel) en fin de master 1, j'ai choisi de travailler sous la tutelle de monsieur LUXEY-BITRI sur le sujet des communications réseaux dites paire à paire.
Avec peu de connaissances en réseaux, j'y ai vu l'occasion de découvrir un domaine qui m'était inconnu et de me confronter à un sujet de recherche pour l'instant peu exploré mais 
avec un potentiel d'application très important. En effet, si aujourd'hui les communications réseaux sont majoritairement centralisées, on voit émerger de plus en plus de réseaux
dits décentralisés. 

\paragraph{}

Les réseaux pair à pair sont des réseaux informatiques dans lesquels chaque client est à la fois client et serveur. C'est à dire que chaque client peut à la fois envoyer et recevoir
des données. Ces réseaux sont décentralisés, c'est à dire qu'il n'y a pas forcément de serveur central qui gère le stockage et la distribution des données. Les clients sont donc
égaux entre eux et peuvent communiquer directement entre eux. 

\paragraph{}

Les réseaux pair à pair sont utilisés dans de nombreux domaines. On peut citer par exemple les réseaux de partage de fichiers, les réseaux de jeux vidéos, les réseaux de communication
entre objets connectés, etc. Avec mon encadrant nous avons choisi de nous intéresser à un type de réseau décentralisé particulier, les réseaux pair à pair dans un contexte de communication 
textuelle temps réel. 


\section{Problématique}

\paragraph{}

Quelles sont les solutions possible pour mettre en place une communication textuelle temps réel entre deux clients dans un réseau pair à pair ?

\section{Objectifs}

\paragraph{}

L'objectif de ce projet est de mettre en place une communication textuelle temps réel entre deux clients dans un réseau pair à pair. Pour cela, nous allons étudier les différentes
solutions existantes et les implémenter. Le but étant d'en apprendre d'avantage sur les réseaux pair à pair, comprendre leur fonctionnement et leurs limites.

\section{Plan}

\paragraph{}

Dans un premier temps, afin de mieux comprendre le fonctionnement des réseaux pair à pair, nous allons étudier les différentes solutions existantes. Nous verrons qu'il existe
deux types de réseaux pair à pair, les réseaux pair à pair structurés et les réseaux pair à pair non structurés. Nous étudierons les avantages et les inconvénients de chacun.
Nous verrons également qu'il existe des solutions hybrides qui combinent les deux types de réseaux.

\paragraph{}

Dans un second temps, nous étudierons les différentes façons de mettre en place une communication textuelle temps réel entre deux clients dans un réseau pair à pair. Nous verrons
qu'il existe plusieurs solutions possibles telles que l'utilisation de sockets, l'utilisation de WebRTC ou encore l'utilisation de WebSockets.

\paragraph{}
Dans un troisième temps, présenterons la l'alternative avec WebRTC que nous avons implémenter. Nous expliquerons les raisons de ce choix et nous détaillerons l'implémentation de
notre solution. Avant de tenter la réalisation de cette même solution via des sockets. Nous verrons dans cette partie les difficultés rencontrées, les solutions apportées et les
limites de notre implémentation. 

\paragraph{}
Enfin, nous conclurons sur les résultats obtenus et nous évoquerons les perspectives d'amélioration de notre solution.

